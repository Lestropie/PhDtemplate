%
% This file is where you should define various abbreviations, acronyms, and other useful definitions.
% Some examples are provided to demonstrate the concept.
%
% Acronyms
% The \newabbrev command can be used with two arguments, the first in {}
% brackets and the second in [] brackets, as below. The contents of the {}
% brackets are printed the first time the command is used in the document; the
% contents of the [] brackets are then printed for all subsequent instances.
% This is handy for fully defining an acronym the first time it appears, without
% having to contantly re-check your document to make sure that your acronyms are
% defined. However MAKE SURE you actually define the acronym form within the {}
% brackets!
\newabbrev \act      {Anatomically-Constrained Tractography (ACT)}[ACT]
\newabbrev \csf      {Cerebro-Spinal Fluid (CSF)}[CSF]
\newabbrev \mri      {Magnetic Resonance Imaging (MRI)}[MRI]
\newabbrev \nmr      {Nuclear Magnetic Resonance (NMR)}[NMR]
\newabbrev \sift     {Spherical-deconvolution Informed Filtering of Tractograms (SIFT)}[SIFT]
%
%
% Terminology - no acronyms
% Sometimes it's handy to be able to write frequently-occurring text in an
% abbreviated form, even though you don't want the text to be abbreviated in
% the thesis itself. I found that using these actually made the text more
% readable within the editor, but it may be a personal preference.
\newabbrev \fe      {frequency-encode}
\newabbrev \fed     {frequency-encode direction}
\newabbrev \fpop    {fibre population}
\newabbrev \fst     {Fourier Shift Theorem}
\newabbrev \gm      {grey matter}
\newabbrev \pe      {phase-encode}
\newabbrev \ped     {phase-encode direction}
\newabbrev \rkfour  {4th-order Runge-Kutta integration}
\newabbrev \roi     {region of interest}
\newabbrev \rois    {regions of interest}
\newabbrev \ssel    {slice-select}
\newabbrev \sseld   {slice-select direction}
\newabbrev \wm      {white matter}
%
%
% Italics
% Certain text should always appear as italic text e.g. latin terms. Rather than 
% manually italicising these every time, commands can be used to do it for you.
% As with the terminology section, I found that these made the text appear more readable
% in the editor, as it meant less {} brackets and \em (emphasis) commands.
\newabbrev \adhoc    {{\em ad hoc}}
\newabbrev \apri     {{\em a priori}}
\newabbrev \bvalue   {{\em b}-value}
\newabbrev \bvalues  {{\em b}-values}
\newabbrev \defacto  {{\em de facto}}
\newabbrev \denovo   {{\em de novo}}
\newabbrev \exvivo   {{\em ex vivo}}
\newabbrev \insilico {{\em in silico}}
\newabbrev \invivo   {{\em in vivo}}
\newabbrev \kspace   {{\em k}-space}
\newabbrev \qspace   {{\em q}-space}
%
%
% You can add anything else to this file you want e.g. software names, institutions, ...
% 
%